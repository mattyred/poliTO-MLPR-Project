%%%%%%%%%%%%%%%%%%%%%%%%%%%%%%%%%%%%%%%%%
% Wenneker Article
% LaTeX Template
% Version 2.0 (28/2/17)
%
% This template was downloaded from:
% http://www.LaTeXTemplates.com
%
% Authors:
% Vel (vel@LaTeXTemplates.com)
% Frits Wenneker
%
% License:
% CC BY-NC-SA 3.0 (http://creativecommons.org/licenses/by-nc-sa/3.0/)
%
%%%%%%%%%%%%%%%%%%%%%%%%%%%%%%%%%%%%%%%%%

%----------------------------------------------------------------------------------------
%	PACKAGES AND OTHER DOCUMENT CONFIGURATIONS
%----------------------------------------------------------------------------------------

\documentclass[10pt, a4paper, twocolumn]{article} % 10pt font size (11 and 12 also possible), A4 paper (letterpaper for US letter) and two column layout (remove for one column)
\usepackage{placeins}
\usepackage{multirow}
\newcommand{\comment}[1]{}
\input{structure.tex} % Specifies the document structure and loads requires packages

%----------------------------------------------------------------------------------------
%	ARTICLE INFORMATION
%----------------------------------------------------------------------------------------

\title{MLPR Exam Project: \\Gender Detection} % The article title

\author{
	\authorstyle{Mattia Rosso [s294711]} % Author
	%\newline\newline % Space before institutions
	%\textsuperscript{1}\institution{Universidad Nacional Autónoma de México, Mexico City, Mexico}\\ % Institution 1
	%\textsuperscript{2}\institution{University of Texas at Austin, Texas, United States of America}\\ % Institution 2
	%\textsuperscript{3}\institution{\texttt{LaTeXTemplates.com}} % Institution 3
}

% Example of a one line author/institution relationship
%\author{\newauthor{John Marston} \newinstitution{Universidad Nacional Autónoma de México, Mexico City, Mexico}}

\date{\today} % Add a date here if you would like one to appear underneath the title block, use \today for the current date, leave empty for no date

%----------------------------------------------------------------------------------------

\begin{document}
\maketitle % Print the title
\thispagestyle{firstpage} % Apply the page style for the first page (no headers and footers)

%----------------------------------------------------------------------------------------
%	ABSTRACT
%----------------------------------------------------------------------------------------

\lettrineabstract{This project is inteded to show a binary classification 
task on a datased made of 12 continuous observations coming from speking embeddings. 
A speaker embedding represents a smal-dimensional, fixed size representation of an utterance.
Features can be seen as points in the m-dimensional embedding space (and the embeddings
have already been computed). This is a task where classes are balanced both in training and
evaluation set}

%----------------------------------------------------------------------------------------
%	ARTICLE CONTENTS
%----------------------------------------------------------------------------------------

\section{Dataset analysis}
\subsection{Training and evaluation sets}
The training set contains:
\begin{itemize}
	\item Training Set: 3000 samples belonging to Male class (Label = 0) and
						3000 samples belonging to Female class (Label = 1).
	\item Evaluation Set: 2000 samples belonging to Male class (Label = 0) and
						  2000 samples belonging to Female class (Label = 1).
\end{itemize}

\subsection{Training set features analysis}
\subsection{Features Statistics}
All the features are contiguous and their main statics can be
showed through a boxplot in figure \ref{boxplot}.
\comment{
\FloatBarrier
	\begin{table}
		\caption{Features Statistics}
		\centering
		\begin{tabular}{ |l|l|l|l|l| }
			\hline
			\multicolumn{5}{ |c| }{Statistics} \\
			\hline
			Feature & Min & Max & Mean & StdDev \\ \hline
			\multirow{2}{*}{0}
			 & 0 & 0 & 0 & 0 \\
			 & 0 & 0 & 0 & 0  \\ \hline
			\multirow{2}{*}{1}
			 & 0 & 0 & 0 & 0  \\
			 & 0 & 0 & 0 & 0  \\ \hline
			\multirow{2}{*}{2}
			& 0 & 0 & 0 & 0  \\
			& 0 & 0 & 0 & 0  \\ \hline
		\end{tabular}
	\end{table}
\FloatBarrier
}
\begin{figure}[ht!]
	\includegraphics[width=\linewidth]{./Pictures/FeaturesAnalysis/boxplot.png}
	\caption{Features boxplot}
	\label{boxplot} 
\end{figure}

\subsection{Features Distribution}
%------------------------------------------------

\subsection{Subsection}

Nam ante risus, tempor nec lacus ac, congue pretium dui. Donec a nisl est. Integer accumsan mauris eu ex venenatis mollis. Aliquam sit amet ipsum laoreet, mollis sem sit amet, pellentesque quam. Aenean auctor diam eget erat venenatis laoreet. In ipsum felis, tristique eu efficitur at, maximus ac urna. Aenean pulvinar eu lorem eget suscipit. Aliquam et lorem erat. Nam fringilla ante risus, eget convallis nunc pellentesque non. Donec ipsum nisl, consectetur in magna eu, hendrerit pulvinar orci. Mauris porta convallis neque, non viverra urna pulvinar ac. Cras non condimentum lectus. Aliquam odio leo, aliquet vitae tellus nec, imperdiet lacinia turpis. Nam ac lectus imperdiet, luctus nibh a, feugiat urna.

\begin{itemize}
	\item First item in a list 
	\item Second item in a list 
	\item Third item in a list
\end{itemize}

Nunc egestas quis leo sed efficitur. Donec placerat, dui vel bibendum bibendum, tortor ligula auctor elit, aliquet pulvinar leo ante nec tellus. Praesent at vulputate libero, sit amet elementum magna. Pellentesque sodales odio eu ex interdum molestie. Suspendisse lacinia, augue quis interdum posuere, dolor ipsum euismod turpis, sed viverra nibh velit eget dolor. Curabitur consectetur tempus lacus, sit amet luctus mauris interdum vel. Curabitur vehicula convallis felis, eget mattis justo rhoncus eget. Pellentesque et semper lectus.

\begin{description}
	\item[First] This is the first item
	\item[Last] This is the last item
\end{description}

Donec nec nibh sagittis, finibus mauris quis, laoreet augue. Maecenas aliquam sem nunc, vel semper urna hendrerit nec. Pellentesque habitant morbi tristique senectus et netus et malesuada fames ac turpis egestas. Maecenas pellentesque dolor lacus, sit amet pretium felis vestibulum finibus. Duis tincidunt sapien faucibus nisi vehicula tincidunt. Donec euismod suscipit ligula a tempor. Aenean a nulla sit amet magna ullamcorper condimentum. Fusce eu velit vitae libero varius condimentum at sed dui.

%------------------------------------------------

\subsection{Subsection}

In hac habitasse platea dictumst. Etiam ac tortor fermentum, ultrices libero gravida, blandit metus. Vivamus sed convallis felis. Cras vel tortor sollicitudin, vestibulum nisi at, pretium justo. Curabitur placerat elit nunc, sed luctus ipsum auctor a. Nulla feugiat quam venenatis nulla imperdiet vulputate non faucibus lorem. Curabitur mollis diam non leo ullamcorper lacinia.

Morbi iaculis posuere arcu, ut scelerisque sem. Class aptent taciti sociosqu ad litora torquent per conubia nostra, per inceptos himenaeos. Mauris placerat urna id enim aliquet, non consequat leo imperdiet. Phasellus at nibh ut tortor hendrerit accumsan. Phasellus sollicitudin luctus sapien, feugiat facilisis risus consectetur eleifend. In quis luctus turpis. Nulla sed tellus libero. Pellentesque metus tortor, convallis at tellus quis, accumsan faucibus nulla. Fusce auctor eleifend volutpat. Maecenas vel faucibus enim. Donec venenatis congue congue. Integer sit amet quam ac est aliquam aliquet. Ut commodo justo sit amet convallis scelerisque.

\begin{enumerate}
	\item First numbered item in a list
	\item Second numbered item in a list
	\item Third numbered item in a list
\end{enumerate}

Aliquam elementum nulla at arcu finibus aliquet. Praesent congue ultrices nisl pretium posuere. Nunc vel nulla hendrerit, ultrices justo ut, ultrices sapien. Duis ut arcu at nunc pellentesque consectetur. Vestibulum eget nisl porta, ultricies orci eget, efficitur tellus. Maecenas rhoncus purus vel mauris tincidunt, et euismod nibh viverra. Mauris ultrices tellus quis ante lobortis gravida. Duis vulputate viverra erat, eu sollicitudin dui. Proin a iaculis massa. Nam at turpis in sem malesuada rhoncus. Aenean tempor risus dui, et ultrices nulla rutrum ut. Nam commodo fermentum purus, eget mattis odio fringilla at. Etiam congue et ipsum sed feugiat. Morbi euismod ut purus et tempus. Etiam est ligula, aliquam eget porttitor ut, auctor in risus. Curabitur at urna id dui lobortis pellentesque.


%------------------------------------------------

\section{Section}

\begin{figure}[htpb!]
	\includegraphics[width=\linewidth]{bear.jpg} % Figure image
	\caption{A majestic grizzly bear} % Figure caption
	\label{bear} % Label for referencing with \ref{bear}
\end{figure}

%----------------------------------------------------------------------------------------
%	BIBLIOGRAPHY
%----------------------------------------------------------------------------------------

\printbibliography[title={Bibliography}] % Print the bibliography, section title in curly brackets

%----------------------------------------------------------------------------------------

\end{document}
